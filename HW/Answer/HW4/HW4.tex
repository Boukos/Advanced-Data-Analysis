\documentclass[letterpaper,11pt]{article}
\usepackage{latexsym}
\usepackage[empty]{fullpage}
\usepackage[usenames,dvipsnames]{color}
\usepackage{verbatim}
\usepackage{hyperref}
\usepackage{framed}
\usepackage{tocloft}
\usepackage{bibentry}
\usepackage{amsmath}
\usepackage{scrextend}
\usepackage{listings}
\usepackage{color}
\usepackage{fancyhdr}
\usepackage{graphicx}

%THIS PORTION IS FOR ADDING PAGE NUMBER
\pagestyle{fancy}
\cfoot{}
\rfoot{\thepage}
\renewcommand{\headrulewidth}{0pt}
%THIS PORTION IS FOR ADDING PAGE NUMBER

\urlstyle{same}
\definecolor{mygrey}{gray}{.85}
\definecolor{mygreylink}{gray}{.30}
\textheight=9.0in
\raggedbottom
\raggedright
\setlength{\tabcolsep}{0in}


%The following part is for inserting codes in LaTeX:
\definecolor{codegreen}{rgb}{0,0.6,0}
\definecolor{codegray}{rgb}{0.5,0.5,0.5}
\definecolor{codepurple}{rgb}{0.58,0,0.82}
\definecolor{backcolour}{rgb}{0.95,0.95,0.92}

\lstdefinestyle{mystyle}{
    backgroundcolor=\color{backcolour},
    commentstyle=\color{codegreen},
    keywordstyle=\color{magenta},
    numberstyle=\tiny\color{codegray},
    stringstyle=\color{codepurple},
    basicstyle=\footnotesize,
    breakatwhitespace=false,
    breaklines=true,
    captionpos=b,
    keepspaces=true,
    numbers=left,
    numbersep=5pt,
    showspaces=false,
    showstringspaces=false,
    showtabs=false,
    tabsize=2
}
\lstset{style=mystyle}
%For inserting codes in LaTeX

\begin{document}

\begin{center}
	\textbf{\Huge{Advanced Data Analysis HW4}}
\end{center}

\begin{center}
	\textsl{Ao Liu, al3472}
\end{center}

\bigbreak
\bigbreak
\bigbreak


%%%%%%%%%%%%%%%%%%%%%%%%%%%%%%%%%%%%%
%%%%%%%%%%%%%%   1   %%%%%%%%%%%%%%%%
%%%%%%%%%%%%%%%%%%%%%%%%%%%%%%%%%%%%%


\begin{addmargin}[-2em]{0em} \large{\textbf{1. }}\end{addmargin}
\textbf{Consider the 3 player game shown below, in which player3’s strategy set is ${M1,M2,M3,M4}$. Each cell only lists one payoff, which represents all three players’ payoffs from that strategy profile. E.g., all three players’ payoff is 8 from $(A, C, M_1)$.}
\bigbreak

\textbf{Let $p \in [0, 1]$ denote the probability of playing $A$ in a mixed strategy for player 1 (so $1 - p$ is the probability of playing B), and $q \in [0, 1]$ denote that of playing $C$ in a mixed strategy for player 2 (so $1 - q$ is the probability of playing D). We can denote a mixed strategy profile for players 1 and 2 by a pair $(p, q)$.}

\begin{addmargin}[-1.1em]{0em} \textbf{(a)}\par\end{addmargin}
  \textbf{Compute the expected payoff for player 3 when she plays some $M_j$ given any mixed strategy profile for the opponents, i.e., give algebraic expressions for $v_3(M_j,p,q)$ for each $j \in {1,2,3,4}$.}\par
\bigbreak
\begin{addmargin}[-0.5em]{0em}
\textbf{Answer: }\end{addmargin}


I expect there's a positive correlation between sales of cigarettes and Age, because the people with higher ages tend to smoke more cigarettes, the proportion of people smoking within young people is relatively small.\par
I expect there's a negative correlation between sales of cigarettes and HS, because people with higher education are more aware of the harm of smoking and are tend to somke less.\par
I expect there's a positive correlation between sales of cigarettes and Income, because with more money, people can afford more cigarettes.\par
I don't expect there's a correlation between sales of cigarettes and Black, because there's no supporting evidence that black people are mode likely to smoke more than others.\par
I expect there's a negative correlation between sales of cigarettes and Female, because there are a lot more men smoking than women, the higher the proportion of the women, the less the proportion of people smoking.\par
I expect there's a negative correlation between sales of cigarettes and Price, the less the price is, the more people can afford the cigarettes.\par


\begin{addmargin}[-1.1em]{0em}
\textbf{(b)}\par\end{addmargin}
  \textbf{What three inequalities must hold if $M_2$ is a best response to the mixed strategy profile $(p,q)$?}\par
\bigbreak
\begin{addmargin}[-0.5em]{0em}
\textbf{Answer: }\end{addmargin}



\begin{lstlisting}
> data = read.table("DATACIGARETTE.txt", header = TRUE)
> mat.data <- data.matrix(data[,2:8])
> cor((mat.data))
\end{lstlisting}
so the pairwise correlation coefficient matrix is:
\begin{lstlisting}
         Age	      HS	      Income	    Black	     Female	    Price	    Sales
  Age	1.00000000	-0.09891626	0.25658098	-0.04033021	0.55303189	0.24775673	0.22655492
  HS	-0.09891626	1.00000000	0.53400534	-0.50171191	-0.41737794	0.05697473	0.06669476
  Income	0.25658098	0.53400534	1.00000000	0.01728756	-0.06882666	0.21455717	0.32606789
  Black	-0.04033021	-0.50171191	0.01728756	1.00000000	0.45089974	-0.14777619	0.18959037
  Female	0.55303189	-0.41737794	-0.06882666	0.45089974	1.00000000	0.02247351	0.14622124
  Price	0.24775673	0.05697473	0.21455717	-0.14777619	0.02247351	1.00000000	-0.30062263
  Sales	0.22655492	0.06669476	0.32606789	0.18959037	0.14622124	-0.30062263	1.00000000
\end{lstlisting}


\begin{lstlisting}
> pairs(data[,2:8])
\end{lstlisting}

The corresponding scatter plot is:
\begin{center} \makebox[\linewidth]{\includegraphics[width=\textwidth]{HW4.jpg}}
\end{center}



\begin{addmargin}[-1.1em]{0em}
\textbf{(c)}\par\end{addmargin}
  \textbf{Show that the three inequalities above are incompatible, using the requirements $p \in [0,1]$ and $q \in [0,1]$. What does this tell you about whether $M_2$ is in the set of rationalizable strategies?}\par
  %by putting the "\par" into the \textbf, error msg will occur...
\bigbreak
\begin{addmargin}[-0.5em]{0em}
\textbf{Answer: }\end{addmargin}

\begin{lstlisting}
> results = lm(Sales ~ Age+HS+Income+Black+Female+Price , data=data)
# install.packages('car')
> library(car)
> vif(results)
\end{lstlisting}
The VIF for 6 variables are:
\begin{lstlisting}
  Age       HS   Income    Black   Female    Price
2.300617 2.676465 2.325164 2.392152 2.406417 1.142181
\end{lstlisting}

\begin{addmargin}[-1.1em]{0em}
\textbf{(d)}\par\end{addmargin}
  \textbf{Argue that no pure strategy is strictly dominated in this game.}\par
  %by putting the "\par" into the \textbf, error msg will occur...
\bigbreak
\begin{addmargin}[-0.5em]{0em}
\textbf{Answer: }\end{addmargin}

Yes, there are outlying Sales observations in the regression model relating Sales to the six predictors.\par
\begin{lstlisting}
> r = rstudent(results)
> data[abs(r)>3,]
\end{lstlisting}

\begin{lstlisting}
  State	Age	HS	Income	Black	Female	Price	Sales
29	NV	27.8	65.2	4563	5.7	49.3	44.0	189.5
30	NH	28.0	57.6	3737	0.3	51.1	34.1	265.7
\end{lstlisting}

As we can see from the result, NV and NH are the two states that have outlying Sales observations in the regression model relating Sales to the six predictors.


\begin{addmargin}[-1.1em]{0em}
\textbf{(e)}\par\end{addmargin}
  \textbf{Combining parts (c) and (d), what has this example proved?}\par
  %by putting the "\par" into the \textbf, error msg will occur...
\bigbreak
\begin{addmargin}[-0.5em]{0em}
\textbf{Answer: }\end{addmargin}


\begin{lstlisting}
> results = lm(Sales ~ Age+HS+Income+Black+Female+Price , data=data)
> lev = hat(model.matrix(results))
> lev
\end{lstlisting}

\begin{lstlisting}
  0.148834504030601 0.580160351053579 0.0496985364502016 0.13709423363853 0.0601062187168606 0.130138939805311 0.174671432351856 0.109964259157845 0.719712840922186 0.2984807799694 0.0992462576906176 0.216903455841289 0.092060813754798 0.078138923121058 0.0863031077553889 0.0600834204236368 0.0600531614367214 0.229271714316211 0.147216983914844 0.0592277100717082 0.082438148474407 0.111768500306865 0.0788600537461464 0.0595424855799236 0.220268293959589 0.0538126742130819 0.07535749988127 0.0713061951148198 0.171156408722539 0.0663450574293963 0.112543589203342 0.205163482615724 0.125210794051503 0.189612219012305 0.0977942068782977 0.0427023659576899 0.0747312407280888 0.194824611487451 0.110848283936785 0.107728709447137 0.157387827636549 0.0702636797470909 0.0861059621640753 0.073209300335295 0.310573704010913 0.0646432821046025 0.121378178796484 0.06441386857486 0.139413729014416 0.0386706972156815 0.0845573052310269
\end{lstlisting}

According to the common rule we will flag any observation whose leverage value satisfies
$$h_{ii} > \frac{2p}{n} = \frac{12}{51}$$

\begin{lstlisting}
> data[which(lev>12/51),1]
> lev[lev>12/51]
\end{lstlisting}

\begin{lstlisting}
AK DC FL UT
0.580160351053579 0.719712840922186 0.2984807799694 0.310573704010913
\end{lstlisting}

Thus, according to the rule above, we get 4 states who have high leverage: AK, DC, FL, UT.

\begin{addmargin}[-1.1em]{0em}
\textbf{(f)}\par\end{addmargin}
  \textbf{Combining parts (c) and (d), what has this example proved?}\par
  %by putting the "\par" into the \textbf, error msg will occur...
\bigbreak
\begin{addmargin}[-0.5em]{0em}
\textbf{Answer: }\end{addmargin}

Typically, points with Cook's distance greater than 1 are classified as being influential, so we calculate Cook's distance for every state's data:
\begin{lstlisting}
> cook = cooks.distance(results)
> cook
\end{lstlisting}
Then we get the 51 Cook's distance:
\begin{lstlisting}
0.000872312260768373 0.11046568368098 0.000231406438616766 0.00533644985289951 0.00087184813943891 0.00487422897991362 0.000903689085754166 0.0247623942663158 0.272772360871267 0.00031718427842994 0.00228486844474409 0.149108047215374 0.00577305260600313 0.000515725336002152 0.000235292422912618 0.0022195814166927 0.0011742764662855 0.0293108038740074 0.0091907941594545 0.0046197884422297 0.00853949173434342 5.43666855436729e-05 0.000993543307494363 0.000416583654710988 0.00357717450931961 0.00181158830548335 0.00222167629931825 0.0106704252779454 0.226884733564416 0.243073931655456 0.00865807771956428 0.00862556323250348 0.0128750489379433 0.0514726373485391 0.000694147592466878 0.000198786969647856 0.000926126973805015 0.000632210801320141 0.00231792501997608 3.5406858359337e-07 0.00559710600247828 0.0021868920733631 0.000167678660476866 0.000276633855805063 0.0861896935150744 0.00441253690931487 0.0144781318896138 0.00554156138259945 0.00241683043673791 0.000216940322227759 6.00182901861485e-05
\end{lstlisting}
No one is greater than 1, so there's no influential points according to the rule of Cook's distance.

\begin{addmargin}[-1.1em]{0em}
\textbf{(g)}\par\end{addmargin}
  \textbf{Combining parts (c) and (d), what has this example proved?}\par
  %by putting the "\par" into the \textbf, error msg will occur...
\bigbreak
\begin{addmargin}[-0.5em]{0em}
\textbf{Answer: }\end{addmargin}

\begin{lstlisting}
> results = lm(log(Sales) ~ Age+HS+Income+Black+Female+Price, data=data)
> r = rstudent(results)
> data[abs(r)>3,]
\end{lstlisting}

\begin{lstlisting}
  State	Age	HS	Income	Black	Female	Price	Sales
29	NV	27.8	65.2	4563	5.7	49.3	44.0	189.5
30	NH	28.0	57.6	3737	0.3	51.1	34.1	265.7
\end{lstlisting}

As we can see from the result, NV and NH are the two states that have outlying Sales observations in the regression model relating Sales to the six predictors.




\begin{lstlisting}
> lev = hat(model.matrix(results))
> lev
\end{lstlisting}

\begin{lstlisting}
  0.148834504030601 0.580160351053579 0.0496985364502016 0.13709423363853 0.0601062187168606 0.130138939805311 0.174671432351856 0.109964259157845 0.719712840922186 0.2984807799694 0.0992462576906176 0.216903455841289 0.092060813754798 0.078138923121058 0.0863031077553889 0.0600834204236368 0.0600531614367214 0.229271714316211 0.147216983914844 0.0592277100717082 0.082438148474407 0.111768500306865 0.0788600537461464 0.0595424855799236 0.220268293959589 0.0538126742130819 0.07535749988127 0.0713061951148198 0.171156408722539 0.0663450574293963 0.112543589203342 0.205163482615724 0.125210794051503 0.189612219012305 0.0977942068782977 0.0427023659576899 0.0747312407280888 0.194824611487451 0.110848283936785 0.107728709447137 0.157387827636549 0.0702636797470909 0.0861059621640753 0.073209300335295 0.310573704010913 0.0646432821046025 0.121378178796484 0.06441386857486 0.139413729014416 0.0386706972156815 0.0845573052310269
\end{lstlisting}

According to the common rule we will flag any observation whose leverage value satisfies
$$h_{ii} > \frac{2p}{n} = \frac{12}{51}$$

\begin{lstlisting}
> data[which(lev>12/51),1]
> lev[lev>12/51]
\end{lstlisting}

\begin{lstlisting}
AK DC FL UT
0.580160351053579 0.719712840922186 0.2984807799694 0.310573704010913
\end{lstlisting}

Thus, according to the rule above, we get 4 states who have high leverage: AK, DC, FL, UT.


Typically, points with Cook's distance greater than 1 are classified as being influential, so we calculate Cook's distance for every state's data:
\begin{lstlisting}
> cook = cooks.distance(results)
> cook
\end{lstlisting}
Then we get the 51 Cook's distance:
\begin{lstlisting}
0.00359045513224516 0.240240855616072 0.00121318532616218 0.00940000348486041 0.000293993996183841 0.000845445364141786 0.00157627217231795 0.0336654337125128 0.14134926250704 4.17359595508581e-08 0.00203727701223271 0.270758042867878 0.00458499785548776 0.000451889520957869 9.31888737750166e-06 0.00184184619189831 0.000560973161470141 0.0319246707978281 0.0204097590400288 0.00921415105332014 0.00810736995024783 7.12278615229615e-05 0.00254520116778151 0.0002981214100366 0.00317957208142376 0.00184483532854467 0.00116873114507587 0.0108404640748746 0.242062092944229 0.161220045757967 0.0123781360184546 0.0103388197355609 0.0176912765990876 0.0540788613509783 0.00205060932036119 4.12952664005199e-05 0.000879330434682632 0.0030802159952135 0.00441873438600284 8.57636871488496e-05 0.005679380081201 0.00398597075671328 0.000654080555535082 0.00040384707347726 0.238154414273449 0.00956732233083818 0.0132426216290245 0.00700748304948474 0.00221365370129141 0.000195642965139178 0.00154531793916012
\end{lstlisting}
No one is greater than 1, so there's no influential points according to the rule of Cook's distance.



%%%%%%%%%%%%%%%%%%%%%%%%%%%%%%%%%%%%%
%%%%%%%%%%%%%%   2   %%%%%%%%%%%%%%%%
%%%%%%%%%%%%%%%%%%%%%%%%%%%%%%%%%%%%%


\begin{addmargin}[-2em]{0em} \large{\textbf{2. }}\end{addmargin}
  \begin{addmargin}[-1.1em]{0em} \textbf{(a) Write down the (mathematical) definition of a pure strategy Nash equilibrium.}\par\end{addmargin}
    \textbf{}\par
  \bigbreak
  \begin{addmargin}[-0.5em]{0em}
  \textbf{Answer: }\end{addmargin}






\begin{addmargin}[-1.1em]{0em}
\textbf{(b)}\par\end{addmargin}
  \textbf{Is every strategy that is part of a pure strategy Nash equilibrium rationalizable?
}\par
   \textbf{}\par
 \bigbreak
 \begin{addmargin}[-0.5em]{0em}
 \textbf{Answer: }\end{addmargin}





\end{document}


%%%%%%%%%%%%%%%%%%%%%%%%%%%%%%%%%%%%%
%%%%%%%%%%%%%%   #   %%%%%%%%%%%%%%%%
%%%%%%%%%%%%%%%%%%%%%%%%%%%%%%%%%%%%%


%Insert pics:
%%%%%%%%%%%%%
%\begin{center}
  %\makebox[\linewidth]{\includegraphics[width=\textwidth]{4640HW6.jpg}}
%\end{center}

%insert a complicated tab...
%%%%%%%%%%%%%%%%%%%%%%%%%%%%
%\begin{center}
%\begin{tabular}{ p{12cm}p{1cm}p{1cm}p{1cm}  }
%& \multicolumn{3}{c}{Posterior Quantiles} \\
%\centering{Quantity of Interest} & 25\% & 50\% & 75\% \\
%\hline
%geometric mean for Blue Earth (no basement), exp($\beta_2)$ &4.1& 5.0& 6.5\\
%geometric mean for Blue Earth County (basement), exp($\beta_1+\beta_2)$ &6.1 &7.1 &8.2\\
%geometric mean for Clay County (no basement), exp($\beta_3)$& 3.8& 4.7 &5.8\\
%geometric mean for Clay County (basement), exp($\beta_1+\beta_3)$ &5.6& 6.5& 7.6\\
%geometric mean for Goodhue County (no basement), exp($\beta_4)$ & 3.9 &4.9& 6.2\\
%geometric mean for Goodhue County (basement), exp($\beta_1+\beta_4)$ &5.8& 6.8& 7.9\\
%factor for basement vs. no basement, exp($\beta_1$)&1.1& 1.4 &1.7\\
%geometric sd of predictions, exp($\sigma$)&2.1 &2.2& 2.4\\
%\end{tabular}
%\end{center}

%%%insert code snippets:
%%%%%%%%%%%%%%%%%%%%%%%%
%\begin{lstlisting}
%INSERT CODE HERE
%\end{lstlisting}

%%insert equation with severl lines:
%\begin{align}
%LEFT &= RIGHT1 \nonumber\\
%     &= RIGHT2 \nonumber\\
%     &= RIGHT3 \nonumber
%\end{align}


%ssh-add ~/.ssh/id_rsa
