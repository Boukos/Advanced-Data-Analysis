\documentclass[letterpaper,11pt]{article}
\usepackage{latexsym}
\usepackage[empty]{fullpage}
\usepackage[usenames,dvipsnames]{color}
\usepackage{verbatim}
\usepackage{hyperref}
\usepackage{framed}
\usepackage{tocloft}
\usepackage{bibentry}
\usepackage{amsmath}
\usepackage{scrextend}
\usepackage{listings}
\usepackage{color}
\usepackage{fancyhdr}
\usepackage{graphicx}

%THIS PORTION IS FOR ADDING PAGE NUMBER
\pagestyle{fancy}
\cfoot{}
\rfoot{\thepage}
\renewcommand{\headrulewidth}{0pt}

%THIS PORTION IS FOR ADDING PAGE NUMBER
\urlstyle{same}
\definecolor{mygrey}{gray}{.85}
\definecolor{mygreylink}{gray}{.30}
\textheight=9.0in
\raggedbottom
\raggedright
\setlength{\tabcolsep}{0in}

%The following part is for inserting codes in LaTeX:
\definecolor{codegreen}{rgb}{0,0.6,0}
\definecolor{codegray}{rgb}{0.5,0.5,0.5}
\definecolor{codepurple}{rgb}{0.58,0,0.82}
\definecolor{backcolour}{rgb}{0.95,0.95,0.92}

\lstdefinestyle{mystyle}{
    backgroundcolor=\color{backcolour},
    commentstyle=\color{codegreen},
    keywordstyle=\color{magenta},
    numberstyle=\tiny\color{codegray},
    stringstyle=\color{codepurple},
    basicstyle=\footnotesize,
    breakatwhitespace=false,
    breaklines=true,
    captionpos=b,
    keepspaces=true,
    numbers=left,
    numbersep=5pt,
    showspaces=false,
    showstringspaces=false,
    showtabs=false,
    tabsize=2}
\lstset{style=mystyle}

% Adjust margins
\usepackage[left=0.9in,top=0.7in,right=0.9in,bottom=0.9in]{geometry}

% For centering elements in the tabular form
% http://bit.ly/2mtyAps
\usepackage{array}
\newcolumntype{P}[1]{>{\centering\arraybackslash}p{#1}}

%%%%%%%%%%%%%%%%%%%%%%%%%%%%%%%%%%%%%
%%%%%%   settings end here   %%%%%%%%
%%%%%%%%%%%%%%%%%%%%%%%%%%%%%%%%%%%%%

\begin{document}

\begin{center}
	\textbf{\Huge{Advanced Data Analysis HW5}}
\end{center}

\begin{center}
	\textsl{Ao Liu, al3472}
\end{center}

\bigbreak
\bigbreak
\bigbreak


%%%%%%%%%%%%%%%%%%%%%%%%%%%%%%%%%%%%%
%%%%%%%%%%%%%%   1   %%%%%%%%%%%%%%%%
%%%%%%%%%%%%%%%%%%%%%%%%%%%%%%%%%%%%%

\begin{addmargin}[-2em]{0em}
  \large{\textbf{1. }}
\end{addmargin}
\textbf{}\par

\begin{addmargin}[-1.1em]{0em}
  \textbf{(a)}\par
\end{addmargin}
\textbf{}\par
\bigbreak
\begin{addmargin}[-0.5em]{0em}
  \textbf{Answer: }
\end{addmargin}

\begin{lstlisting}
data = read.csv("Shuttle.csv", header = TRUE)
glm(ThermalDistress~Temperature, data = data, family = binomial("logit"))
\end{lstlisting}

\begin{lstlisting}
  Call:  glm(formula = ThermalDistress ~ Temperature, family = binomial("logit"),
      data = data)

  Coefficients:
  (Intercept)  Temperature
      15.0429      -0.2322

  Degrees of Freedom: 22 Total (i.e. Null);  21 Residual
  Null Deviance:	    28.27
  Residual Deviance: 20.32 	AIC: 24.32
\end{lstlisting}


\begin{addmargin}[-1.1em]{0em}
  \textbf{(b)}\par
\end{addmargin}
\textbf{}\par
\bigbreak
\begin{addmargin}[-0.5em]{0em}
  \textbf{Answer: }
\end{addmargin}

According to the result that we got in (a), our estimation of $\beta_1$, the effect of temperature on the probability of thermal distress is $$-0.2322$$
This implies that when we increase the temperature by 1 degree, the odds of having Thermal Distress changes by a multiplicative factor of $e^{-0.2322}$

\begin{addmargin}[-1.1em]{0em}
  \textbf{(c)}\par
\end{addmargin}
\textbf{}\par
\bigbreak
\begin{addmargin}[-0.5em]{0em}
  \textbf{Answer: }
\end{addmargin}


\begin{lstlisting}
confint(glm(ThermalDistress~Temperature, data = data, family = binomial("logit")))
\end{lstlisting}

\begin{lstlisting}
              2.5\%	      97.5\%
(Intercept)	3.3305848	  34.34215133
Temperature	-0.5154718	-0.06082076
\end{lstlisting}
According to the results in R, the $95\%$ confidence interval for $\beta_1$ is $$(-0.515718,-0.06082076)$$
so the the $95\%$ confidence interval for $e^{\beta_1}$ is $$(0.597071743167396,0.940991888047314)$$
This indicates that we are $95\%$ confident that when we increase the temperature by 1 degree, the odds of having Thermal Distress changes by a multiplicative factor between 0.597071743167396 and 0.940991888047314.


\begin{addmargin}[-1.1em]{0em}
  \textbf{(d)}\par
\end{addmargin}
\textbf{}\par
\bigbreak
\begin{addmargin}[-0.5em]{0em}
  \textbf{Answer: }
\end{addmargin}



\end{document}

%%%%%%%%%%%%%%%%%%%%%%%%%%%%%%%%%%%%%
%%%%%%%%%%%%%%   #   %%%%%%%%%%%%%%%%
%%%%%%%%%%%%%%%%%%%%%%%%%%%%%%%%%%%%%

%Insert pics:
%%%%%%%%%%%%%
%\begin{center}
  %\makebox[\linewidth]{\includegraphics[width=\textwidth]{4640HW6.jpg}}
%\end{center}

%insert a complicated tab...
%%%%%%%%%%%%%%%%%%%%%%%%%%%%
%\begin{center}
%\begin{tabular}{ P{12cm}P{1cm}P{1cm}P{1cm}  }
%& \multicolumn{3}{c}{Posterior Quantiles} \\
%\centering{Quantity of Interest} & 25\% & 50\% & 75\% \\
%\hline
%geometric mean for Blue Earth (no basement), exp($\beta_2)$ &4.1& 5.0& 6.5\\
%geometric mean for Blue Earth County (basement), exp($\beta_1+\beta_2)$ &6.1 &7.1 &8.2\\
%geometric mean for Clay County (no basement), exp($\beta_3)$& 3.8& 4.7 &5.8\\
%geometric mean for Clay County (basement), exp($\beta_1+\beta_3)$ &5.6& 6.5& 7.6\\
%geometric mean for Goodhue County (no basement), exp($\beta_4)$ & 3.9 &4.9& 6.2\\
%geometric mean for Goodhue County (basement), exp($\beta_1+\beta_4)$ &5.8& 6.8& 7.9\\
%factor for basement vs. no basement, exp($\beta_1$)&1.1& 1.4 &1.7\\
%geometric sd of predictions, exp($\sigma$)&2.1 &2.2& 2.4\\
%\end{tabular}
%\end{center}

% simple version:
% The capital P is defined in the header
% \begin{center}
% \begin{tabular}{ P{1cm}P{1cm}P{1cm}P{1cm}P{1cm}P{1cm}}
% {} & 1 & 2 & 3 & 4 & 5 \\
% \hline
% 1 &1,1 &0,0 &0,0 &0,0 &0,0\\
% 2 &0,0 &1,1 &0,0 &0,0 &0,0\\
% 3 &0,0 &0,0 &1,1 &0,0 &0,0\\
% 4 &0,0 &0,0 &0,0 &1,1 &0,0\\
% 5 &0,0 &0,0 &0,0 &0,0 &1,1\\
% \end{tabular}
% \end{center}

%%%insert code snippets:
%%%%%%%%%%%%%%%%%%%%%%%%
%\begin{lstlisting}
%INSERT CODE HERE
%\end{lstlisting}

%%insert equation with severl lines:
%\begin{align}
%LEFT &= RIGHT1 \nonumber\\
%     &= RIGHT2 \nonumber\\
%     &= RIGHT3 \nonumber
%\end{align}

%insert a 大括号...
% $$ leftside = \begin{cases}
%   case1 & detail1 \\
%   case2 & detail2 \\
%   case3 & detail3
% \end{cases}$$

% insert text under equation mode:
% \textrm{the text that you need}

% inset a space:
% \:

%ssh-add ~/.ssh/id_rsa
