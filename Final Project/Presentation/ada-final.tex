%%%%%%%%%%%%%%%%%%%%%%%%%%%%%%%%%%%%%%%%%%%%%%%%%%%%%%%%%%%%%%%%%%%%%%
% https://www.overleaf.com/9239625sdykfpbkgqxv#/33299142/
% Overleaf (WriteLaTeX) Example: Molecular Chemistry Presentation
%
% Source: http://www.overleaf.com
%
% In these slides we show how Overleaf can be used with standard
% chemistry packages to easily create professional presentations.
%
% Feel free to distribute this example, but please keep the referral
% to overleaf.com
%
%%%%%%%%%%%%%%%%%%%%%%%%%%%%%%%%%%%%%%%%%%%%%%%%%%%%%%%%%%%%%%%%%%%%%%
% How to use Overleaf:
%
% You edit the source code here on the left, and the preview on the
% right shows you the result within a few seconds.
%
% Bookmark this page and share the URL with your co-authors. They can
% edit at the same time!
%
% You can upload figures, bibliographies, custom classes and
% styles using the files menu.
%
% If you're new to LaTeX, the wikibook is a great place to start:
% http://en.wikibooks.org/wiki/LaTeX
%
%%%%%%%%%%%%%%%%%%%%%%%%%%%%%%%%%%%%%%%%%%%%%%%%%%%%%%%%%%%%%%%%%%%%%%

\documentclass{beamer}

% For more themes, color themes and font themes, see:
% http://deic.uab.es/~iblanes/beamer_gallery/index_by_theme.html
%
\mode<presentation>
{
  \usetheme{Madrid}       % or try default, Darmstadt, Warsaw, ...
  \usecolortheme{default} % or try albatross, beaver, crane, ...
  \usefonttheme{serif}    % or try default, structurebold, ...
  \setbeamertemplate{navigation symbols}{}
  \setbeamertemplate{caption}[numbered]
}

\usepackage[english]{babel}
\usepackage[utf8x]{inputenc}
\usepackage{chemfig}
\usepackage[version=3]{mhchem}

% On Overleaf, these lines give you sharper preview images.
% You might want to `comment them out before you export, though.
\usepackage{pgfpages}
\pgfpagesuselayout{resize to}[%
  physical paper width=8in, physical paper height=6in]

% Here's where the presentation starts, with the info for the title slide
\title[Columbia University]{Analysis of U.S. Regional Crime Rates}
\author{Ziwei Meng, Ao Liu}
%\institute{}
\date{\today}

\begin{document}

\begin{frame}
  \titlepage
\end{frame}

% These three lines create an automatically generated table of contents.
\begin{frame}{Outline}
  \tableofcontents
\end{frame}
\section{Overview}
\subsection{Goal and Procedure}
\begin{frame}{Goal and Procedure}
\begin{itemize}
\item Using the data given to create a regression model.
\item Based on the model, give suggestions on the reduction of the number of serious crimes in their county.
\item Further thoughts about the findings.
\end{itemize}
\end{frame}


\section{Model Building}
\subsection{Data Overview}
\begin{frame}[fragile]
\frametitle{Data Overview}
\begin{itemize}
\item \textbf{Geographic Data:} Land Area, Geographic Region
\item \textbf{Demographic Data:} Total population, Percent of population aged 18-34, Percent Bachelor’s Degree
\item \textbf{Economics Data:} Percent Below Poverty Level, Total Personal Income, Per Capita Income
\end{itemize}
\end{frame}

\subsection{Data Processing}
\begin{frame}[fragile]
\frametitle{Data Processing}
\begin{itemize}
\item Check for missing values (and substitute them with mean values)
\item Calculate more variables that cater to our needs:\\
(1) Population Density = $\frac{Population}{Area}$\\
(2) Physician Per 1000 Population = $\frac{Physician}{Population/1000}$\\
(3) Hospital Beds Per 1000 Population = $\frac{Hospital Beds}{Population/1000}$\\
(4) Crime Rate Per 1000 Population = $\frac{Crimes}{Population/1000}$
\item Randomly Select 330 rows of data to train the regression model, and the remaining 110 rows are used for testing the accuracy of our model
\end{itemize}
\end{frame}


\begin{frame}[fragile]
\frametitle{Heatmap}
First we explore the correlation of variables:
\begin{center}
  \makebox[\linewidth]{\includegraphics[width=5in]{pics/ada1.jpg}}
\end{center}
\end{frame}

\begin{frame}[fragile]
\frametitle{Heatmap}
\begin{itemize}
\item Given 16 predictor variables, some of them are strongly correlated with each other, which will cause us to get some potentially false conclusion, thus we remove these variables.
\item The remaining variables are:\\
\textsl{Area, Percentage of Young People, Percentage of Old People, Percentage of High School, Percentage of Bachelor, Percentage of Poor, Unemployment, Income,  Region, Population Density, Physician Per 1000 Population, Beds Per 1000 Population}
\end{itemize}
\end{frame}


\subsection{Regression Model}
\begin{frame}[fragile]
\frametitle{Regression Model}
\begin{itemize}
\item Given the fact that crime rate is a count value, in this question we fit the data to \textbf{Poisson Regression Model}, to reduce the effect of region size, we add offset to the model, and also use quasi-likelihood in order to prevent over dispersion.
\end{itemize}
\end{frame}


\begin{frame}[fragile]
\frametitle{Regression Model}
\begin{itemize}
\item Then we do the significant test for each variable.\\
\item Through the resulting output table from Poisson Regression, the following variables are insignificant:\\
\textsl{area, percent of old people, percent of people with high school education.}
\item After removing the insignificant variables, we build the Poisson Regression Model again using only the most important variables.
\end{itemize}
\end{frame}


\begin{frame}[fragile]
\frametitle{Outliers}
Check outliers
\begin{center}
  \makebox[\linewidth]{\includegraphics[width=3.2in]{pics/ada2.jpg}}
\end{center}
\end{frame}


\subsection{Interpretation of Variables Effect on Crime Rate}
\begin{frame}[fragile]
\frametitle{Interpretation of Variables Effect on Crime Rate}
Here we interpret the meaning of each parameters in our model:\\
\begin{itemize}
\item percent young:
If we decrease the percent of young people by 1 unit while holding all other variables the same, the crime rate would decrease by a multiplicative factor of 1.017847 on average.\\
\item percent poor: If we decrease the percent on poor people by 1 unit while holding all other variables the same, the crime rate would decrease by a multiplicative factor of 1.024423 on average\\
\item population density: If we decrease the log of the population density by 1 unit while holding all other variables the same, the crime rate would decrease by a multiplicative factor of 1.085662 on average.\\
\end{itemize}
\end{frame}



\begin{frame}[fragile]
\frametitle{Interpretation of Variables Effect on Crime Rate}
\begin{itemize}
\item region: Holding all other variables the same, the crime rate in NC is higher than that in NE by a multiplicative factor of 1.347162 on average, the crime rate in S is higher than that in NE by a multiplicative factor of 1.775532 on average, the crime rate in W is higher than that in NE by a multiplicative factor of 1.673471 on average.\\
\item beds per 1000 population: If we decrease the density of beds per 1000 people by 1 unit while holding all other variables the same, the crime rate would decrease by a multiplicative factor of 1.049475 on average.
\end{itemize}
\end{frame}


\subsection{Prediction}
\begin{frame}[fragile]
\frametitle{Prediction on Testing Data}
\begin{itemize}
\item Finally we use the testing data to predict the crime rate of the remaining 110 counties and examine the accuracy of the regression model
\end{itemize}
\begin{center}
  \makebox[\linewidth]{\includegraphics[width=3.2in]{pics/ada3.jpg}}
\end{center}
\end{frame}


\begin{frame}[fragile]
\frametitle{XGboost Model}
To further explore the data, we fit our data into XGboost Model:
\begin{center}
  \makebox[\linewidth]{\includegraphics[width=3in]{pics/ada4.jpg}}
\end{center}
\end{frame}




\subsection{Discoveries from the Testing Result}
\begin{frame}{Discoveries from the Testing Result}
\begin{itemize}
\item The two models both fit the data well
\item Only several points are outliers, no matter which model we use, so there are some others reasons for their high crime rate that we don't know the exactly.
\item Since our client - Kings County is also among the several outliers in both models, we have to do further analysis to find out the hidden reason for its high crime rate. Otherwise, our suggestions may not be applicable to Kings County.
\end{itemize}

\end{frame}



\begin{frame}[fragile]
\frametitle{With and Without Kings County}

 \begin{center}
 \begin{tabular}{ p{3cm}p{2cm}p{2cm}}
 {} &Poisson &GBM \\
 \hline
with Kings  &24.55210 &22.41788\\
without Kings &16.27706 &15.24004\\
 \end{tabular}
 \end{center}

 \bigbreak
 \bigbreak
 
 
Building the two kinds of models with and without taking King's County into consideration, we can see a big difference in the error, meaning that we have to analyze King's County and other counties separately.
\end{frame}



\begin{frame}[fragile]
\frametitle{Most Important Variables without Kings County}
Let's see what are the most important variables using XGBoost without Kings County:
\begin{center}
  \makebox[\linewidth]{\includegraphics[width=3.5in]{pics/ada5.jpg}}
\end{center}
 From the chart above we can see that for most of the counties in the U.S., the variables that matter most are percent of poor people, region size, number of physicians per 1000 population and so on.
\end{frame}




\begin{frame}[fragile]
\frametitle{Kings County vs other counties in U.S.}
Let's see what are the variables with which Kings County differs most from the others. We can see that among the several variables we focus on, Kings county has a significant high population density, which might be one of the reasons why Kings County has a high crime rate.
\begin{center}
  \makebox[\linewidth]{\includegraphics[width=3.5in]{pics/ada6.jpg}}
\end{center}
\end{frame}


\begin{frame}[fragile]
\frametitle{Kings County vs other counties in NY}
Assuming counties that are adjacent to each other might have more similarities, we see the difference between Kings County and other counties in the State of New York.
\begin{center}
  \makebox[\linewidth]{\includegraphics[width=3.5in]{pics/ada7.jpg}}
\end{center}
We can see that the biggest difference is still population density. Thus, we analyze population density's impact on crime rate.
\end{frame}

\begin{frame}[fragile]
\frametitle{Population Density}
\begin{center}
  \makebox[\linewidth]{\includegraphics[width=5in]{pics/ada10.jpg}}
\end{center}
\end{frame}



\section{Suggestions and Further Thoughts}
\begin{frame}{Suggestions}
Based on the value of the parameters, we give the following suggestions to the officials of Kings County:
\begin{itemize}
\item General suggestions:\\
1. Adopt better policy to raise the income of people.\\
2. Invest more money on education
\item Specific for Kings County:\\
- Control the population of King's County:\\
It is harder to reduce the population, so we increase the land area: land filling.
\end{itemize}

\end{frame}

\begin{frame}{Further Thoughts}
Though we have found out the relationship between high population density and high crime rate in King's County, we want to know why.

\begin{block}{Social Economic Reasons}
"Crime rates spiked in the 1980s and early 1990s as \textbf{the crack epidemic} hit the city."\\
\bigbreak
\small{Crime in New York City - Wikipedia\\ \url{http://bit.ly/2oYXTQQ}}
\end{block}
% The LaTeX wikibook is also a good source of info, e.g.
% http://en.wikibooks.org/wiki/LaTeX/Chemical_Graphics

\begin{block}{Food For Thought}
"New York City Crime in the Nineties - The New Yoker"\\
\bigbreak
\small{\url{http://bit.ly/2os9ZTQ}}
\end{block}

\end{frame}


\end{document}
%\url{www.overleaf.com/help}




ggplot(data = melted_cormat, aes(Var2, Var1, fill = value))+
 geom_tile(color = "white")+
 scale_fill_gradient2(low = "blue", high = "red", mid = "white",
   midpoint = 0, limit = c(-1,1), space = "Lab",
   name="Pearson\nCorrelation") +
  theme_minimal()+
 theme(axis.text.x = element_text(angle = 45, vjust = 1,
    size = 12, hjust = 1), axis.title.x=element_blank(), axis.title.y=element_blank())+
 coord_fixed()

\subsection{Goal and Procedure}
\begin{frame}{Goal and Procedure}
\begin{itemize}
\item Using the data given to create a regression model.
\item Based on the model, give suggestions on the reduction of the number of serious crimes in their county.
\item Further thoughts about the findings.
\end{itemize}
\end{frame}


\section{Model Building}
\subsection{Data Overview}
\begin{frame}[fragile]
\frametitle{Data Overview}
\begin{itemize}
\item \textbf{Geographic Data:} Land Area, Geographic Region
\item \textbf{Demographic Data:} Total population, Percent of population aged 18-34, Percent Bachelor’s Degree
\item \textbf{Economics Data:} Percent Below Poverty Level, Total Personal Income, Per Capita Income
\end{itemize}
\end{frame}

\subsection{Data Processing}
\begin{frame}[fragile]
\frametitle{Data Processing}
\begin{itemize}
\item Check for missing values (and substitute them with mean values)
\item Calculate more variables that cater to our needs:\\
(1) Population Density = $\frac{Population}{Area}$\\
(2) Physician Per 1000 Population = $\frac{Physician}{Population/1000}$\\
(3) Hospital Beds Per 1000 Population = $\frac{Hospital Beds}{Population/1000}$\\
(4) Crime Rate Per 1000 Population = $\frac{Crimes}{Population/1000}$
\item Randomly Select 330 rows of data to train the regression model, and the remaining 110 rows are used for testing the accuracy of our model
\end{itemize}
\end{frame}


\begin{frame}[fragile]
\frametitle{Heatmap}
First we explore the correlation of variables:
\begin{center}
  \makebox[\linewidth]{\includegraphics[width=5in]{pics/pics/ada1.jpg}}
\end{center}
\end{frame}

\begin{frame}[fragile]
\frametitle{Heatmap}
\begin{itemize}
\item Given 16 predictor variables, some of them are strongly correlated with each other, which will cause us to get some potentially false conclusion, thus we remove these variables.
\item The remaining variables are:\\
\textsl{Area, Percentage of Young People, Percentage of Old People, Percentage of High School, Percentage of Bachelor, Percentage of Poor, Unemployment, Income,  Region, Population Density, Physician Per 1000 Population, Beds Per 1000 Population}
\end{itemize}
\end{frame}


\subsection{Regression Model}
\begin{frame}[fragile]
\frametitle{Regression Model}
\begin{itemize}
\item Given the fact that crime rate is a count value, using an ordinary linear regression model will affect model's accuracy, in this question we fit the data to \textbf{Poisson Regression Model}, to reduce the effect of region area size, we add offset to the model, and also quasi-likelihood in order to prevent over dispersion.
\end{itemize}

\end{frame}



\begin{frame}[fragile]
\frametitle{Regression Model}
\begin{itemize}
\item Then we do the significant test for each variable.\\
\item Through the resulting output table from Poisson Regression, the following variables are insignificant:\\
\textsl{area, percent of old people, percent of people with high school education.}
\item After removing the insignificant variables, we build the Poisson Regression Model again using only the most important variables.
\end{itemize}
\end{frame}


\begin{frame}[fragile]
\frametitle{Outliers}
Check outliers
\begin{center}
  \makebox[\linewidth]{\includegraphics[width=3.2in]{pics/pics/ada2.jpg}}
\end{center}
\end{frame}


\subsection{Interpretation of Variables Effect on Crime Rate}
\begin{frame}[fragile]
\frametitle{Interpretation of Variables Effect on Crime Rate}
Here we interpret the meaning of each parameters in our model:\\
\begin{itemize}
\item percent young:
If we decrease the percent of young people by 1 unit while holding all other variables the same, the crime rate would decrease by a multiplicative factor of 1.017847 on average.\\
\item percent poor: If we decrease the percent on poor people by 1 unit while holding all other variables the same, the crime rate would decrease by a multiplicative factor of 1.024423 on average\\
\item population density: If we decrease the log of the population density by 1 unit while holding all other variables the same, the crime rate would decrease by a multiplicative factor of 1.085662 on average.\\
\end{itemize}
\end{frame}



\begin{frame}[fragile]
\frametitle{Interpretation of Variables Effect on Crime Rate}
\begin{itemize}
\item region: Holding all other variables the same, the crime rate in NC is higher than that in NE by a multiplicative factor of 1.347162 on average, the crime rate in S is higher than that in NE by a multiplicative factor of 1.775532 on average, the crime rate in NC is higher than that in W by a multiplicative factor of  on average.\\
\item beds per 1000 population: If we decrease the density of beds per 1000 people by 1 unit while holding all other variables the same, the crime rate would decrease by a multiplicative factor of 1.049475 on average.
\end{itemize}
\end{frame}


\subsection{Prediction}
\begin{frame}[fragile]
\frametitle{Prediction on Testing Data}
\begin{itemize}
\item Finally we use the testing data to predict the crime rate of the remaining 110 counties and examine the accuracy of the regression model
\end{itemize}
\begin{center}
  \makebox[\linewidth]{\includegraphics[width=3.2in]{pics/pics/ada3.jpg}}
\end{center}
\end{frame}


\begin{frame}[fragile]
\frametitle{XGboost Model}
To further explore the data, we fit our data into XGboost Model:
\begin{center}
  \makebox[\linewidth]{\includegraphics[width=3in]{pics/pics/ada4.jpg}}
\end{center}
\end{frame}


\subsection{Discoveries from the Testing Result}
\begin{frame}{Discoveries from the Testing Result}
\begin{itemize}
\item The two models both fit the data well
\item Only several points are outliers, no matter which model we use, so there are some others reasons for their high crime rate that we don't know the exactly.
\item Since our client - Kings County is also among the several outliers in both models, we have to do further analysis to find out the hidden reason for its high crime rate. Otherwise, our suggestions may not be applicable to Kings County.
\end{itemize}

\end{frame}

\begin{frame}[fragile]
\frametitle{Most Important Variables}
Let's see what are the most important variables using XGBoost:
\begin{center}
  \makebox[\linewidth]{\includegraphics[width=4in]{pics/pics/ada5.jpg}}
\end{center}
Top variables: percent of poor people (which is harder), region area, pyhsician, etc.
\end{frame}

\begin{frame}[fragile]
\frametitle{Kings County vs other counties in U.S.}
Let's see what are the variables with which Kings County differs most from the others. We can see that among the several variables we focus on, Kings county has a significant high population density, which might be one of the reasons why Kings County has a high crime rate.
\begin{center}
  \makebox[\linewidth]{\includegraphics[width=3.5in]{pics/pics/ada6.jpg}}
\end{center}
\end{frame}

\begin{frame}[fragile]
\frametitle{Kings County vs other counties in NY}
Assuming counties that are adjacent to each other might have more similarities, we see the difference between Kings County and other counties in the State of New York.
\begin{center}
  \makebox[\linewidth]{\includegraphics[width=3.5in]{pics/pics/ada7.jpg}}
\end{center}
We can see that the biggest difference is still population density. Thus, we analyze population density's impact on crime rate.
\end{frame}

\begin{frame}[fragile]
\frametitle{Population Density}
\begin{center}
  \makebox[\linewidth]{\includegraphics[width=5in]{pics/pics/ada10.jpg}}
\end{center}
\end{frame}


\section{Suggestions and Further Thoughts}
\begin{frame}{Suggestions}
Based on the value of the parameters, we give the following suggestions to the officials of Kings County:
\begin{itemize}
\item General suggestions:\\
1. Adopt better policy to raise the income of people.\\
2. Invest more money on education
\item Specific for Kings County:\\
- Control the population density of King's County:\\
Instead of decreasing the population, we can increase the land area: land filling and so on.
\end{itemize}

\end{frame}

\begin{frame}{Further Thoughts}
Though we have found out the relationship between high population density and high crime rate in King's County, we want to know why.

\begin{block}{Social Economic Reasons}
"Crime rates spiked in the 1980s and early 1990s as \textbf{the crack epidemic} hit the city."\\
\bigbreak
\small{Crime in New York City - Wikipedia\\ \url{http://bit.ly/2oYXTQQ}}
\end{block}
% The LaTeX wikibook is also a good source of info, e.g.
% http://en.wikibooks.org/wiki/LaTeX/Chemical_Graphics

\begin{block}{Food For Thought}
"New York City Crime in the Nineties - The New Yoker"\\
\bigbreak
\small{\url{http://bit.ly/2os9ZTQ}}
\end{block}

\end{frame}


\end{document}
%\url{www.overleaf.com/help}


% New York Leads Big Cities in Robbery Rate, but Drops in Murders
% http://www.nytimes.com/1991/08/11/nyregion/new-york-leads-big-cities-in-robbery-rate-but-drops-in-murders.html

% New York Killings Set a Record, While Other Crimes Fell in 1990
% http://www.nytimes.com/1991/04/23/nyregion/new-york-killings-set-a-record-while-other-crimes-fell-in-1990.html

% MURDERS SOAR IN 21 PRECINCTS IN NEW YORK
% http://www.nytimes.com/1987/03/24/nyregion/murders-soar-in-21-precincts-in-new-york.html

% Uniform Crime Reports and Index of Crime in Kings in the State of Californi enforced by Kings County from 1985 to 2005
% http://www.disastercenter.com/californ/crime/916.htm

% New York Crime Rates 1960 - 2015
% http://www.disastercenter.com/crime/nycrime.htm
 