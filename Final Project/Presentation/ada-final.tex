%%%%%%%%%%%%%%%%%%%%%%%%%%%%%%%%%%%%%%%%%%%%%%%%%%%%%%%%%%%%%%%%%%%%%%
% https://www.overleaf.com/9239625sdykfpbkgqxv#/33299142/
% Overleaf (WriteLaTeX) Example: Molecular Chemistry Presentation
%
% Source: http://www.overleaf.com
%
% In these slides we show how Overleaf can be used with standard
% chemistry packages to easily create professional presentations.
%
% Feel free to distribute this example, but please keep the referral
% to overleaf.com
%
%%%%%%%%%%%%%%%%%%%%%%%%%%%%%%%%%%%%%%%%%%%%%%%%%%%%%%%%%%%%%%%%%%%%%%
% How to use Overleaf:
%
% You edit the source code here on the left, and the preview on the
% right shows you the result within a few seconds.
%
% Bookmark this page and share the URL with your co-authors. They can
% edit at the same time!
%
% You can upload figures, bibliographies, custom classes and
% styles using the files menu.
%
% If you're new to LaTeX, the wikibook is a great place to start:
% http://en.wikibooks.org/wiki/LaTeX
%
%%%%%%%%%%%%%%%%%%%%%%%%%%%%%%%%%%%%%%%%%%%%%%%%%%%%%%%%%%%%%%%%%%%%%%

\documentclass{beamer}

% For more themes, color themes and font themes, see:
% http://deic.uab.es/~iblanes/beamer_gallery/index_by_theme.html
%
\mode<presentation>
{
  \usetheme{Madrid}       % or try default, Darmstadt, Warsaw, ...
  \usecolortheme{default} % or try albatross, beaver, crane, ...
  \usefonttheme{serif}    % or try default, structurebold, ...
  \setbeamertemplate{navigation symbols}{}
  \setbeamertemplate{caption}[numbered]
}

\usepackage[english]{babel}
\usepackage[utf8x]{inputenc}
\usepackage{chemfig}
\usepackage[version=3]{mhchem}

% On Overleaf, these lines give you sharper preview images.
% You might want to `comment them out before you export, though.
\usepackage{pgfpages}
\pgfpagesuselayout{resize to}[%
  physical paper width=8in, physical paper height=6in]

% Here's where the presentation starts, with the info for the title slide
\title[Columbia University]{Analysis of U.S. Regional Crime Rates}
\author{Ziwei Meng, Ao Liu}
%\institute{}
\date{\today}

\begin{document}

\begin{frame}
  \titlepage
\end{frame}

% These three lines create an automatically generated table of contents.
\begin{frame}{Outline}
  \tableofcontents
\end{frame}

\section{Overview}

\begin{frame}{Overview}

\end{frame}

\subsection{Goal and Procedure}
\begin{frame}{Goal and Procedure}

\begin{itemize}
\item implement policies that will lead to the reduction of the number serious crimes in their county
\item compute the regression model based on the training set and test the accuracy of the model using the test data.
\end{itemize}

\begin{block}{The \texttt{mhchem} package}
The \texttt{mhchem} package provides simple commands for typesetting chemical molecular formulae and equations. Created by Martin Hensel, a detailed user guide can be found here:\\[0.4cm]
\small{\url{http://mirror.ox.ac.uk/sites/ctan.org/macros/latex/contrib/mhchem/mhchem.pdf}}
\end{block}
% The LaTeX wikibook is also a good source of info, e.g.
% http://en.wikibooks.org/wiki/LaTeX/Chemical_Graphics

\end{frame}

\section{Model Building}

\subsection{Data Overview}
\begin{frame}[fragile]
\frametitle{Data Overview}

\begin{itemize}
\item Geological Data
\item Demographic Data
\item Economics Data
\end{itemize}

\end{frame}

\subsection{Data Processing}
\begin{frame}[fragile]
\frametitle{Data Processing}

\begin{itemize}
\item
\item Check invalid values (and remove them)
\item
\end{itemize}



\end{frame}

\subsection{Regression Model}
\begin{frame}[fragile]
\frametitle{Regression Model}

empty page

\end{frame}

\subsection{Interpretation of Parameters and Other Results}
\begin{frame}[fragile]
\frametitle{Interpretation of Parameters and Other Results}

empty page

\end{frame}

\section{Suggestions and Improvements}
\begin{frame}{Suggestions\dots{}}

\begin{itemize}
\item 1
\item 2
\item 3
\end{itemize}

\end{frame}

\end{document}
%\url{www.overleaf.com/help}
